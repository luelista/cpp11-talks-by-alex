\PassOptionsToPackage{unicode}{hyperref}
\documentclass{beamer}
\usepackage[utf8]{inputenc}
\usepackage{lmodern} % more font symbols
\usepackage{perpage}
\usepackage{pfnote}
\usepackage{transparent}

\beamertemplatenavigationsymbolsempty

\MakePerPage{footnote}

\newcommand{\emptyline}[1][1]{\vspace{#1\baselineskip}}
\newcommand{\hint}[1]{{\transparent{0.6}#1}}
\newcommand{\rarrow}[0]{$\rightarrow$ }

\setbeamercovered{transparent}
\setbeamertemplate{footline}[frame number]

\usepackage{caption}
\usepackage[T1]{fontenc} %proper tt brace
\usepackage{listings}
\usepackage{lmodern} % more font symbols
\usepackage{relsize}
\usepackage{xcolor}

\newcommand{\std}[1]{\lstinline|std::#1|}
\newcommand{\boost}[1]{\lstinline|boost::#1|}

%c from texinfo.tex
\def\ifmonospace{\ifdim\fontdimen3\font=0pt }

%c C plus plus
\def\C++{%
\ifmonospace%
	C++%
\else%
	C\kern-.033em\raise.30ex\hbox{\smaller\textbf{++}}%
\fi%
\spacefactor1000 }

\lstset{
	language=C++,
	columns=flexible, %possibly better-looking text
	keepspaces=true,
	showstringspaces=false,
	%gobble=12, %default-indent (document, frame, listing) * tabsize = 3*4 = 12
	tabsize=4,
	basicstyle=\footnotesize\ttfamily,
	commentstyle=\color[rgb]{0,0.6,0},
	identifierstyle=\color[rgb]{0,0,0},
	keywordstyle=\color[rgb]{0,0,1},
	stringstyle=\color[rgb]{0.58,0,0.92},
	emphstyle=\itshape,
	backgroundcolor=\color[rgb]{0.92,0.92,0.92},
	escapeinside={/\%}{\%/}, %include TeX via /%\texhere%/
	morekeywords={alignas,alignof,auto,char16_t,char32_t,constexpr,decltype,default,final,noexcept,nullptr,override,static_assert,thread_local},
	literate=%
		*
		{;}{{{\color[rgb]{0.07,0.07,0.07};}}}1
		{,}{{{\color[rgb]{0.07,0.07,0.07},}}}1
		{.}{{{\color[rgb]{0.07,0.07,0.07}.}}}1
		{(}{{{\color[rgb]{0.2,0.2,0.2}(}}}1
		{)}{{{\color[rgb]{0.2,0.2,0.2})}}}1
		{[}{{{\color[rgb]{0.2,0.2,0.2}[}}}1
		{]}{{{\color[rgb]{0.2,0.2,0.2}]}}}1
		{>}{{{\color[rgb]{0.2,0.2,0.2}>}}}1
		{<}{{{\color[rgb]{0.2,0.2,0.2}<}}}1
		{\{}{{{\color[rgb]{0.2,0.2,0.2}\{}}}1
		{\}}{{{\color[rgb]{0.2,0.2,0.2}\}}}}1
}

\captionsetup[lstlisting]{justification=justified,singlelinecheck=false}


\author{Alexander Hirsch}
\begin{document}
	\title{Dealing with pointers in modern \C++}
	\subtitle{\insertauthor}
	
	\begin{frame}
		\begin{center}
			{\Huge Dealing with pointers in \\ modern \C++}
		\end{center}
		
		\only<beamer:0>{
			\emptyline[5.5]
			\begin{center}
				\raisebox{-.35\height}{
					\href{https://creativecommons.org/publicdomain/zero/1.0/}{\includegraphics[scale=0.5]{cc-zero.pdf}}
				}%
				\hint{$\leftarrow$ link}
				
				\emptyline[0.5]
				To the extent possible under law, Alexander Hirsch has waived all copyright and related or neighboring rights to this handout.
			\end{center}
		}
	\end{frame}
	
	\begin{frame}{Who am I?}
		\begin{itemize}
			\item Alexander Hirsch
				
			\leftskip=2em
				\item E-Mail: 1zeeky@gmail.com
				\item Jabber/XMPP: z33ky@jabber.ccc.de
				
			\leftskip=0pt
			
			\emptyline
			\item Computer Science student at TU Darmstadt
			\item hobby-programmer
			\emptyline
			\item looked into many languages and programming paradigms
			\item most experience with \C++ (started getting proficient around 2007)
		\end{itemize}
	\end{frame}
		
	\begin{frame}{Disclaimer}
		What I state as facts might actually be untrue (though they are to my knowledge).
		
		\emptyline
		
		What I present as something ``you should do'', might not actually me a common idiom
		and just my personal opinion.
		
		\pause
		
		\emptyline
		Examples are not necessarily in good coding style. They are just intended to show specific features and hint at use cases.
		
		\only<beamer:0>{
			\emptyline
			They should be valid \C++, apart from missing \lstinline|\#include|s and some code should be inside a function, albeit it not being shown this way.
			
			Except of course where noted that the example would produce an error.
		}
	\end{frame}
	
	\begin{frame}{A shout out}
		A shout out to these websites I find very helpful for getting information about \C++:
		\begin{itemize}
			\item \url{https://en.wikipedia.org/wiki/C\%2B\%2B11}
			\item \url{http://cppreference.com}
			\item \url{https://cppandbeyond.com}
			\item \url{https://stackoverflow.com}
			\item \url{https://isocpp.org}
			\item \url{http://meetingcpp.com}
		\end{itemize}
		
		\pause
		
		\emptyline
		The Standard document is obviously also a very good source. \\
		The (free) draft is mostly fine, too.
	\end{frame}
	
	\begin{frame}{Dealing with pointers in modern \C++}
		\begin{small}
			\tableofcontents
		\end{small}
	\end{frame}
	
	\section{Introduction: Pointers}
	\begin{frame}
		\begin{center}
			{\Huge Introduction: Pointers}
			
			\emptyline
			{\Large what, why and how}
		\end{center}
	\end{frame}
	
	\begin{frame}<1>[label=SaH]{Stack \& Heap}
		\begin{itemize}
			\item Stack \rarrow program stack
			\begin{itemize}
				\item local variables
				\item function arguments
				\item return address
			\end{itemize}
			
			\pause
			
			\item Heap \rarrow ``memory pool''
			\begin{itemize}
				\item dynamic lifetime
				\item large data-structures \hint{(prevent stack overflow)}
				\item objects, whose size is only known at run-time \hint{(data hiding, polymorphism)}
			\end{itemize}
		\end{itemize}
	\end{frame}
	
	%close enough
	\lstdefinelanguage[ARMv8]{Assembler}[x86masm]{Assembler}{
		comment=[l]{;},
		literate=%
			*
			{,}{{{\color[rgb]{0.07,0.07,0.07},}}}1
			{=}{{{\color[rgb]{0.2,0.2,0.2}=}}}1
			{\{}{{{\color[rgb]{0.2,0.2,0.2}\{}}}1
			{\}}{{{\color[rgb]{0.2,0.2,0.2}\}}}}1
	}
	\begin{frame}<handout:0>[fragile=singleslide]{Stack \& Heap - stack in assembly}
		\begin{lstlisting}[language={[ARMv8]Assembler},gobble=12]
			;save work registers and lr
			;lr is Link Register (return address)
			push {r9-r10,lr}
			
			;note: many arguments are passed via register
			ldr r0, =fmt
			mov r1, 2
			bl printf
			
			;restore work registers and lr
			pop {r9-r10,lr}
		\end{lstlisting}
	\end{frame}
	
	\againframe<2|handout:0>{SaH}
	
	\begin{frame}{Pointers}
		Many people seem to have trouble \textbf{understanding what pointers are}.
		
		\begin{center}
			\includegraphics[scale=0.74]{pointer_img.pdf}
		\end{center}
		
	\end{frame}
	
	\begin{frame}{Why do we need pointers?}
		\only<1>{
			Stack addressing done via offset to frame-pointer (done by compiler).
		}
		\only<2-|handout:0>{
			\hint{Stack addressing done via offset to frame-pointer (done by compiler).}
		}
		
		\onslide<2->{
			\rarrow Heap addressing
		}
		
		\pause
		
		Other use-cases:
		\begin{itemize}
			\item optional arguments/return values
			\item returning or passing big objects as arguments without copy
		\end{itemize}
	\end{frame}
	
	\begin{frame}[fragile]{Allocating objects}
		\begin{lstlisting}[title=Stack]
			int  a;
			char b[4];
			a = 42;
			b[0] = '0';
		\end{lstlisting}
		
		\hint{detail: Arrays implemented as pointers to the first element}
		\hint{  \rarrow \lstinline|a[i]| is syntactic sugar for \lstinline|*(a + i)|}
		
		\pause
		
		\begin{lstlisting}[title=Heap]
			int  *a   = new int;
			char  b[] = new char[4];
			//wrong: a = 42; -> set address to 42
			//       C++'s type system would also disallow this
			//       without explicit casting
			*a = 42;
			//note: same as in stack example
			b[0] = '0';
		\end{lstlisting}
			
		\lstinline|int| and \lstinline|char[]| on the Heap, pointers on the Stack
	\end{frame}
	
	\begin{frame}[fragile]{Deallocating objects}
		\begin{lstlisting}[title=Stack]
			{
			    int  a;
			    char b[4];
			    //a and b implicitly popped off the stack
			}
		\end{lstlisting}
		
		\pause
		
		\begin{lstlisting}[title=Heap]
			{
			    int  *a   = new int;
			    char  b[] = new char[4];
			    delete   a;
			    delete[] b;
			    //a and b implicitly popped off the stack
			}
		\end{lstlisting}
	\end{frame}
	
	\begin{frame}{The problem with pointers}
		So, what's the \textbf{problem} with pointers? \\
		How do we need to ``deal'' with them?
		
		\pause
		
		\begin{itemize}
			\item don't forget to \textbf{delete} your pointer
			\begin{itemize}
				\item operator delete?
				\item free(void*)?
				\item DestroyObject(Object*)?
			\end{itemize}
			\pause
			\item but also only do it \textbf{once}
			\pause
			\item delete your pointer in \textbf{all paths}, also when exceptions are thrown
		\end{itemize}
	\end{frame}
		
	\begin{frame}{The problem with pointers}
		Am I even responsible for deleting the pointer?
		% hide first <+->
		\only<+->{}
		\begin{itemize}
			\item \lstinline|SDL_Window* SDL_CreateWindow(...)|\footnote{SDL 2.0.3, \url{http://www.libsdl.org}} \\
				\rarrow \only<+->{you can: \lstinline|SDL_DestroyWindow|}
				
			\item \lstinline|char** PHYSFS_enumerateFiles(...)|\footnote{PhysicsFS 2.0.3, \url{https://icculus.org/physicsfs/}} \\
				\rarrow \only<+->{yes: \lstinline|PHYSFS_freeList|}
				
			\item \lstinline|GList* g_list_reverse(...)|\footnote{GLib 2.38.2, \url{https://developer.gnome.org/glib}} \\
				\rarrow \only<+->{kind of: \lstinline|g_list_reverse| operates in-place}
				
			\item \lstinline|char* strdup(...)|\footnote{standard C library (C99)} \\
				\rarrow \only<+->{yes, \lstinline|free|}
				
			\item \lstinline|char* strtok(...)|\footnotemark[4] \\
				\rarrow \only<+->{no, it uses the memory you passed in}
				
		\end{itemize}
		% workaround bug: footnotecounter resets
		\only<0->{}
	\end{frame}
	
	\section{Avoiding Pointers}
	\begin{frame}
		\begin{center}
			{\Huge Avoiding Pointers}
			
			\emptyline
			{\Large Avoid dealing with pointers \\ by avoiding them in the first place}
		\end{center}
	\end{frame}
	
	\begin{frame}{Avoiding pointers - avoid \lstinline|container<T*>|}
		I've seen containers like \std{vector<T*>} or \std{list<T*>} for no reason. \\
		\std{vector<T>} and \std{list<T>} usually work just as well.
		
		\pause
		
		\emptyline
		Reasons you really do want a \lstinline|T*| container:
		\begin{itemize}
			\item polymorphism \hint{(consider intrusive lists \rarrow less indirection)}
			\item need to copy the elements, but \lstinline|T| is non-copyable or very expensive to copy \hint{(consider Ranges\footnote{e.g. Boost\footnotemark.Range})}
		\footnotetext{reputable, peer-reviewed collection of \C++ libraries for general and various specific use cases}
			\item container of references to existing objects
		\end{itemize}
		
		\pause
		
		\emptyline
		Tip: \boost{ptr_vector} (or \boost{ptr_list}, or ...); handles deletion, supports (customizable) deep-copies
	\end{frame}
	
	\begin{frame}[fragile]{Avoiding pointers - lvalue-references}
		A reference is an alias to another object.
		
		Compared to pointers:
		\begin{itemize}
			\item cannot be reassigned
				\only<beamer:0>{(assignment assigns to variable, not reference)}
			\item there is no ``\lstinline|nullptr|-reference''
			\item semantics: you refer to an object (but don't own it)
		\end{itemize}
		
		\pause
		
		\begin{lstlisting}
			void byCopy(int  a) {  a = 1; }
			void byPtr (int *a) { *a = 2; }
			void byRef (int &a) {  a = 3; }
			int main() {
			    int i = 0;
			    byCopy(i); //i still 0
			    byPtr(&i); //i now 2
			    byRef(i);  //i now 3
			}
		\end{lstlisting}
		
		\pause
		
		Note: \lstinline|T&|-containers (also \lstinline|T&|-arrays) are impossible.
		
		\leftskip=2em Consider \std{reference_wrapper}.
	\end{frame}
	
	\begin{frame}{Avoiding pointers - rvalue-references vs lvalue-references}
		\begin{tabular}{ll}
			lvalue:                               & non-temporary object \\
			                                      & \lstinline|x| in \lstinline|int x;| \\
			\action<2->{(p\footnote{pure})rvalue} & \action<2->{a temporary} \\
			                                      & \action<2->{the result of \lstinline|x + y|}
		\end{tabular}
		
		\pause
		%another one to compensate for the \actions
		\pause
		
		\emptyline[2]
		With lvalue-references, you create \\
		an \textbf{alias for something already named}.
		
		\pause
		
		\emptyline
		With rvalue-references, you create \\
		a \textbf{name for something unnamed}.
	\end{frame}
	
	\begin{frame}{Avoiding pointers - rvalue-references}
		When you take a rvalue-reference you \textbf{extend the lifetime} of the temporary.
		
		That also means, that a rvalue-reference is a lvalue.
		
		\pause
		
		When the rvalue-reference goes out of scope, the destructor of the referenced object is called.
		
		\only<beamer:0>{
			\hint{This is not the case for lvalue-references, for which it would be disastrous since the reference is just an alias to something existing somewhere else.}
		}
	\end{frame}
	
	\begin{frame}{Avoiding pointers - rvalue-references}
		When we copy an object, we must \textbf{duplicate all its data} (think copy-constructor).
		
		\pause
		
		\emptyline
		If we knew that the object we're copying is a \textbf{temporary} that won't be used anymore, we could \textbf{cannibalize} it.
		
		\pause
		
		\emptyline
		An rvalue-reference tells us that it references a temporary.
	\end{frame}
	
	\begin{frame}{Avoiding pointers - move constructor}
		A constructor taking an rvalue-reference \only<beamer:0>{of the type that is constructed} is called \textbf{move-constructor}.
		
		\pause
		
		\emptyline
		It is called that, because the process of making a rvalue from a lvalue is called \textbf{moving}.
		
		\pause
		
		\emptyline[2]
		The utility function \lstinline|std::move| was created for this.
	\end{frame}
	
	\begin{frame}[fragile=singleslide]{Avoiding pointers - rvalue-references examples}
		\begin{lstlisting}[gobble=12]
			struct BigMatrix
			{
			    //std::vector already supports rvalue-references,
			    //that wouldn't be interesting
			    int *Data;
			};
			
			//move constructor taking a rvalue-reference as argument
			BigMatrix::BigMatrix(BigMatrix &&that)
			     //no need to allocate our own memory
			    :Data(that.Data)
			{
			    //make sure that.~BigMatrix() doesn't free our data!
			    that.Data = nullptr;
			}
			
			BigMatrix::~BigMatrix()
			{
			    delete[] Data;
			}
		\end{lstlisting}
	\end{frame}
	
	\begin{frame}[fragile=singleslide]{Avoiding pointers - rvalue-references examples cont'}
		\begin{lstlisting}[gobble=12]
			//don't get the idea to return a rvalue-reference
			BigMatrix operator*(const BigMatrix &a, const BigMatrix &b)
			{
			    BigMatrix result;
			    ...
			    //std::move makes a rvalue-reference
			    //from a lvalue-reference
			    return std::move(result);
			}
			
			int main()
			{
			    BigMatrix a(...), b(...);
			    //move-constructed (or not with compiler optimization)
			    BigMatrix c(a * b);
			    //will not be move (or copy) constructed, uses temporary
			    BigMatrix &&d = a * b;
			}
		\end{lstlisting}
	\end{frame}
	
	\begin{frame}{Avoiding pointers - avoid C-style arrays}
		Prefer \std{array} or \std{vector} to C-style arrays (\lstinline|T array[size]; T *array = new T[size];|)
		
		\begin{itemize}
			\item \std{array} \rarrow compile-time known sizes \\ \hint{beware of stack-overflow}
			\item \std{vector} \rarrow dynamic sizable
		\end{itemize}
		
		\pause
		
		\emptyline
		Raw access (\lstinline|T*|):
		\begin{itemize}
			\item via \lstinline|data()|
			\item elements guaranteed to be stored continuously \rarrow interface with C-style APIs
		\end{itemize}
	\end{frame}
	
	\begin{frame}<beamer:0>{Avoiding pointers - avoid C-style arrays - benefits}
		You can easily switch the underlying data-structure and usually not worry about anything more.
		
		\hint{E.g. vector, list, deque}
		
		\emptyline
		You can get bounds-checking.
		
		\emptyline
		STL containers provide a convenient interface for various operations, including
		\begin{itemize}
			\item adding and removing elements
			\item resizing or pre-allocating (reserving) memory
			\item copying the container
		\end{itemize}
		
		Some containers won't provide the functions that don't make sense for them (e.g. resizing \std{array}).
		
		The shared interface is defined in various \C++ Container concepts.
	\end{frame}
	
	\begin{frame}<beamer:0>{Avoiding pointers - avoid C-style arrays - benefits}
		Algorithms designed for usage with the container-interface also implemented by \std{vector}, like
		\begin{itemize}
			\item sorting
			\item set operations
			\item transformations
			\item creating partial copies
		\end{itemize}
		
		\emptyline
		You can use the range-based \lstinline|for| loop.
		
		Note: Stack allocated C-style arrays work too, \\
		but heap allocated ones do not.
	\end{frame}
	
	\begin{frame}[fragile=singleslide]{Avoiding pointers - avoid C-style arrays examples}
		\begin{lstlisting}
			std::vector<int> iv = { 1, 2, 3, 4, 5 };
			int *ip = iv.data();
			assert(ip[0] == 1);
			++ip[0];
			assert(iv[0] == 2);
			
			std::vector<int> cv(ip, ip + 5);
			assert(std::equal(iv, cv));
			assert(iv.data() != cv.data());
			
			std::vector<int> mv(std::move(iv));
			assert(mv.data() == ip);
			assert(iv.size() == 0);
		\end{lstlisting}
	\end{frame}
	
	\begin{frame}{Avoiding pointers - (Array TS\footnote{Technical Specification}) \std{dynarray}}
		Post \C++14 \std{dynarray}:
		\begin{itemize}
			\item size set at run-time
			\item (unlike \std{vector}) don't grow or shrink after that
			\item either on stack or heap (cannot specify manually)
		\end{itemize}
		
		Provides similar interface to \std{array} and \std{vector} for raw access to continuous data.
	\end{frame}
	
	\begin{frame}{Avoiding pointers - \boost{optional}}
		\boost{optional}\footnote{\std{optional} in Library Fundamentals TS}
		\begin{itemize}
			\item like \lstinline[morekeywords={option}]|option| in Standard ML and OCaml
			\item or like \lstinline[morekeywords={Maybe}]|Maybe| in Haskell
		\end{itemize}
		
		\begin{itemize}
			\item \boost{optional} either has a value or is uninitialized
			\pause
			\item uninitialized state is a well defined state (like a \lstinline|nullptr|)
			      invalid access leads to undefined behavior
		\end{itemize}
		
		\pause
		
		\emptyline
		Usage for optional arguments and return values.
	\end{frame}
	
	\begin{frame}[fragile=singleslide]{Avoiding pointers - \boost{optional}}
		\begin{lstlisting}[gobble=12]
			boost::optional<int> a;
			assert(!a);
			a = 5;
			assert(a && *a == 5);
			a.reset(); //same as a = boost::none;
			assert(!a);
		\end{lstlisting}
	\end{frame}
	
	\section{Smart pointers}
	\begin{frame}
		\begin{center}
			{\Huge Smart pointers}
			
			\emptyline
			{\Large Wrapping pointers in a safer shell}
		\end{center}
	\end{frame}
	
	\begin{frame}{Smart pointers}
		Smart pointers are classes that \textbf{wrap around a pointer}-member.
		
		By carrying \textbf{ownership-semantics}, they enable implicit release of resources (like memory).
		
		\pause
		
		\emptyline
		\C++98 has \std{auto_ptr}. I've rarely seen it in the wild.
		
		\pause
		
		\emptyline
		\C++TR1 suggested \std{shared_ptr} and \std{weak_ptr}. They got in \C++11.
		
		\pause
		
		\emptyline
		\C++11 replaced \std{auto_ptr} with \std{unique_ptr}%
		\only<beamer:0>{, which makes use of new language features}.
		
		\pause
		
		\emptyline[1.5]
		
		Boost has \boost{intrusive_ptr}.
	\end{frame}
	
	\begin{frame}{\std{shared_ptr}}
		\std{shared_ptr} is a \textbf{reference-counting} pointer.
		
		Only reference-counting is thread-safe.
		
		\pause
		
		\begin{itemize}
			\item start with count == 1
			\item when copied, ++count
			\item when destroyed, -{}-count
			\item when count == 0, delete pointer ``automatically''
		\end{itemize}
		
		\only<beamer:0>{
			Default delete via \lstinline|operator delete| (or \lstinline|operator delete[]| for arrays), \\
			but you can specify a custom deleter.
		}
	\end{frame}
	
	\begin{frame}[fragile]{\std{shared_ptr} examples}
		\begin{lstlisting}
			//make_shared creates a shared_ptr
			auto i = std::make_shared<int>(5);
			assert(i.unique() && i.use_count() == 1);
			auto j = i;
			assert(!i.unique() && i.use_count() == 2);
			assert(i == j);
			auto k = std::make_shared<int>(*i);
			i.swap(k);
			assert(i.unique());
			assert(j == k);
		\end{lstlisting}
		
		\pause
		
		\begin{lstlisting}
			auto window = std::shared_ptr<SDL_Window>(
			    SDL_CreateWindow(...), &SDL_DestroyWindow);
			//shared_ptr::get() obtains a raw pointer
			SDL_GetWindowID(window.get());
			//shared_ptr::reset deletes the pointer
			window.reset();
			assert(!window);
		\end{lstlisting}
		
		\pause
		
		\begin{lstlisting}
			auto dup = std::shared_ptr<const char>(strdup(...), &free);
		\end{lstlisting}
	\end{frame}
	
	\begin{frame}{\std{weak_ptr}}
		The \std{weak_ptr} is a \textbf{weak reference} to a \std{shared_ptr}.
		
		\begin{itemize}
			\item does not modify reference-count
			\item still refers to an object
			\pause
			\item in contrast to raw pointer: can know if object was freed (by \std{shared_ptr})
		\end{itemize}
		
		\pause
		
		\emptyline
		To operate on an object a \std{weak_ptr} is pointing to, you must create a \std{shared_ptr} from it.
		
		\only<beamer:0>{
			This way you have the guarantee, that the object remains valid during your operation.
		}
	\end{frame}
	
	\begin{frame}[fragile=singleslide]{\std{weak_ptr} examples}
		\begin{lstlisting}[gobble=12]
			auto i = std::make_shared<int>(23);
			auto weak_i = std::weak_ptr<int>(i);
			
			//weak_ptr can also check the reference-count
			assert(weak_i.use_count() == 1);
			//lock() obtains a shared_ptr
			assert(weak_i.lock() != nullptr);
			
			*weak_i.lock() *= -1;
			assert(*i == -23);
			
			i.reset();
			assert(weak_i.use_count() == 0
			    && weak_i.expired()
			    && !weak_i.lock());
		\end{lstlisting}
	\end{frame}
	
	\begin{frame}{\std{make_shared} and \std{allocate_shared}}
		\std{make_shared} is \textbf{more efficient} than using the \std{shared_ptr(T*)} constructor, \\
		because it can allocate the reference-count right next to the \lstinline|T|.
		
		\pause
		
		\std{make_shared} uses \lstinline|operator new| and \lstinline|operator delete|, \\
		but you can supply a \textbf{custom Allocator}\footnote{template-class with functions to allocate memory and optionally customizable construction and destruction behavior} with \std{allocate_shared}.
		
		\pause
		
		\emptyline
		
		Note that when the \std{shared_ptr(T*)} constructor is used, the object is destroyed when the \textbf{last \lstinline[basicstyle=\bf\ttfamily]|std::shared_ptr| pointing to it} gets destroyed, \\
	while \std{make_shared} (and \std{allocate_shared}) also require \textbf{no \lstinline[basicstyle=\bf\ttfamily]|std::weak_ptr|} to be referring to it.
	\end{frame}
	
	\begin{frame}{\std{unique_ptr}}
		A \std{unique_ptr} is (should be) the \textbf{only owner} of an object.
		
		\begin{itemize}
			\item cannot be copied
			\item \textbf{can} be moved
			\item object freed when pointer goes out of scope
		\end{itemize}
		
		\only<beamer:0>{
			One can optionally supply a custom deleter, like with \std{shared_ptr}.
		}
	\end{frame}
	
	\begin{frame}[fragile]{\std{unique_ptr} examples}
		\begin{lstlisting}
			//note: make_unique is coming in C++14;
			//missing in C++11 (library defect)
			auto i = std::unique_ptr<int>(new int(5));
			//auto j = i; //not possible (cannot copy)
			auto j = std::move(i);
			assert(*j == 5 && !i);
		\end{lstlisting}
		
		\pause
		
		\begin{lstlisting}
			auto window = std::unique_ptr<SDL_Window>(
			    SDL_CreateWindow(...), &SDL_DestroyWindow);
			//unique_ptr::get() obtains a raw pointer
			SDL_GetWindowID(window.get());
			//unique_ptr::reset deletes the pointer
			window.reset();
			//window no longer manages an object,
			//so the following assertion holds
			assert(!window);
		\end{lstlisting}
		
		\pause
		
		\begin{lstlisting}
			auto dup = std::unique_ptr<char*>(strdup("abcdefg", &free);
			//unique_ptr::release releases ownership
			free(dup.release());
			assert(!dup);
		\end{lstlisting}
	\end{frame}
	
	\begin{frame}{\boost{intrusive_ptr}}
		A \boost{intrusive_ptr} ties smart pointer interface into \textbf{existing reference-counting} mechanisms.
		
		\begin{itemize}
			\item basic workings like \std{shared_ptr}
			\item doesn't provide its own reference-count,
			\item provides hooks instead
		\end{itemize}
		
		\emptyline
		This is mostly for interfacing with embedded reference counting.
		
		\only<beamer:0>{
			Note that intrusive reference counting can have performance and memory usage advantages compared to just using \std{shared_ptr}.
		}
	\end{frame}
	
	\begin{frame}[fragile=singleslide]{\boost{intrusive_ptr} examples}
		\begin{lstlisting}[gobble=12]
			//automatically picked up by Boost, selected via signature
			void intrusive_ptr_add_ref(GMappedFile *p)
			{
			    g_mapped_file_ref(p);
			}
			void intrusive_ptr_release(GMappedFile *p)
			{
			    g_mapped_file_unref(p);
			}
			
			//...
			
			//false -> don't increase ref count (default true)
			auto file = boost::intrusive_ptr<GMappedFile>(
			    g_mapped_file_open(...), false);
			auto file_copy = file;
			file.reset();
			assert(!file && file_copy);
		\end{lstlisting}
		
		\only<beamer:0>{
			The \boost{intrusive_ptr} does not introduce thread-safety. \\
			Hooks are not called with a \lstinline|nullptr|.
		}
	\end{frame}
	
	\begin{frame}[fragile]{\lstinline|std::smart_ptr(new T())| - beware of memory leak}
		\begin{lstlisting}
			f(std::smart_ptr<T>(new T()), g());
		\end{lstlisting}
		
		\pause
		
		Compiler might generate something equivalent to this:
		
		\begin{lstlisting}
			auto temp0 = new T();
			auto temp1 = g();
			auto temp2 = std::smart_ptr<T>(temp0);
			
			f(temp2, temp1);
		\end{lstlisting}
		
		\pause
		
		If \lstinline|g()| throws an exception, you have a memory leak.
		
		Using \std{make_shared} or \std{make_unique} would solve this problem.
	\end{frame}
	
	\begin{frame}{Recap}
		\begin{itemize}
			\item Avoiding pointers
			\begin{itemize}
				\item \std{array}, \std{vector} and \std{dynarray}\textsuperscript{(Array TS)} instead of C-style arrays
				\item avoid \lstinline|container<T*>|: store plain \lstinline|T|, \\
					consider \boost{ptr_container} or intrusive lists
				\item references instead of pointers to avoid copies
				\item \boost{optional} (\std{optional}\textsuperscript{(Library Fundamentals TS)}) for optional values (instead of \lstinline|nullptr|)
			\end{itemize}
			\emptyline
			\item Smart pointers
			\begin{itemize}
				\item \std{shared_ptr} and \std{weak_ptr} - shared ownership
				\item \std{unique_ptr} - single ownership
				\item \boost{intrusive_ptr} - wrap existing reference-counting
			\end{itemize}
		\end{itemize}
	\end{frame}
	
	\begin{frame}<handout:0>{The End}
		\begin{center}
			\emptyline[2]
			{\Huge Thank you for tuning in.}
			
			\emptyline
			{\LARGE Questions?}
			
			\emptyline[10]
			\hint{{\large {\ttfamily :3}}}
			
			\hint{{\large *rawr*}}
		\end{center}
	\end{frame}

\end{document}
