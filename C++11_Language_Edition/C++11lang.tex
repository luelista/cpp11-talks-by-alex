\PassOptionsToPackage{unicode}{hyperref}
\documentclass{beamer}
\usepackage[utf8]{inputenc}
\usepackage{lmodern} % more font symbols
\usepackage{perpage}
\usepackage{pfnote}
\usepackage{transparent}

\beamertemplatenavigationsymbolsempty

\MakePerPage{footnote}

\newcommand{\emptyline}[1][1]{\vspace{#1\baselineskip}}
\newcommand{\hint}[1]{{\transparent{0.6}#1}}
\newcommand{\rarrow}[0]{$\rightarrow$ }

\setbeamercovered{transparent}
\setbeamertemplate{footline}[frame number]

\usepackage{caption}
\usepackage[T1]{fontenc} %proper tt brace
\usepackage{listings}
\usepackage{lmodern} % more font symbols
\usepackage{relsize}
\usepackage{xcolor}

\newcommand{\std}[1]{\lstinline|std::#1|}
\newcommand{\boost}[1]{\lstinline|boost::#1|}

%c from texinfo.tex
\def\ifmonospace{\ifdim\fontdimen3\font=0pt }

%c C plus plus
\def\C++{%
\ifmonospace%
	C++%
\else%
	C\kern-.033em\raise.30ex\hbox{\smaller\textbf{++}}%
\fi%
\spacefactor1000 }

\lstset{
	language=C++,
	columns=flexible, %possibly better-looking text
	keepspaces=true,
	showstringspaces=false,
	%gobble=12, %default-indent (document, frame, listing) * tabsize = 3*4 = 12
	tabsize=4,
	basicstyle=\footnotesize\ttfamily,
	commentstyle=\color[rgb]{0,0.6,0},
	identifierstyle=\color[rgb]{0,0,0},
	keywordstyle=\color[rgb]{0,0,1},
	stringstyle=\color[rgb]{0.58,0,0.92},
	emphstyle=\itshape,
	backgroundcolor=\color[rgb]{0.92,0.92,0.92},
	escapeinside={/\%}{\%/}, %include TeX via /%\texhere%/
	morekeywords={alignas,alignof,auto,char16_t,char32_t,constexpr,decltype,default,final,noexcept,nullptr,override,static_assert,thread_local},
	literate=%
		*
		{;}{{{\color[rgb]{0.07,0.07,0.07};}}}1
		{,}{{{\color[rgb]{0.07,0.07,0.07},}}}1
		{.}{{{\color[rgb]{0.07,0.07,0.07}.}}}1
		{(}{{{\color[rgb]{0.2,0.2,0.2}(}}}1
		{)}{{{\color[rgb]{0.2,0.2,0.2})}}}1
		{[}{{{\color[rgb]{0.2,0.2,0.2}[}}}1
		{]}{{{\color[rgb]{0.2,0.2,0.2}]}}}1
		{>}{{{\color[rgb]{0.2,0.2,0.2}>}}}1
		{<}{{{\color[rgb]{0.2,0.2,0.2}<}}}1
		{\{}{{{\color[rgb]{0.2,0.2,0.2}\{}}}1
		{\}}{{{\color[rgb]{0.2,0.2,0.2}\}}}}1
}

\captionsetup[lstlisting]{justification=justified,singlelinecheck=false}


\author{Alexander Hirsch}
\begin{document}
	\title{Cool, new features of \C++11}
	\subtitle{\insertauthor}
	
	\begin{frame}
		\begin{center}
			{\Huge  Cool, new features of \C++11 \\}
			{\LARGE (Pt. 1: Language Version)}
			
			\emptyline
			{\Large An incomplete(!) introduction to changes introduced in \C++11}
		\end{center}
		
		\only<beamer:0>{
			\emptyline[5.5]
			\begin{center}
				\raisebox{-.35\height}{
					\href{https://creativecommons.org/publicdomain/zero/1.0/}{\includegraphics[scale=0.5]{cc-zero.pdf}}
				}%
				\hint{$\leftarrow$ link}
				
				\emptyline[0.5]
				To the extent possible under law, Alexander Hirsch has waived all copyright and related or neighboring rights to this handout.
			\end{center}
		}
	\end{frame}
	
	\begin{frame}{Who am I?}
		\begin{itemize}
			\item Alexander Hirsch
				
			\leftskip=2em
				\item E-Mail: 1zeeky@gmail.com
				\item Jabber/XMPP: z33ky@jabber.ccc.de
				
			\leftskip=0pt
			
			\emptyline
			\item Computer Science student at TU Darmstadt
			\item hobby-programmer
			\emptyline
			\item looked into many languages and programming paradigms
			\item most experience with \C++ (started getting proficient around 2007)
		\end{itemize}
	\end{frame}
		
	\begin{frame}{Disclaimer}
		What I state as facts might actually be untrue (though they are to my knowledge).
		
		\emptyline
		
		What I present as something ``you should do'', might not actually me a common idiom
		and just my personal opinion.
		
		\pause
		
		\emptyline
		Examples are not necessarily in good coding style. They are just intended to show specific features and hint at use cases.
		
		\only<beamer:0>{
			\emptyline
			They should be valid \C++, apart from missing \lstinline|\#include|s and some code should be inside a function, albeit it not being shown this way.
			
			Except of course where noted that the example would produce an error.
		}
	\end{frame}
	
	\begin{frame}{A shout out}
		A shout out to these websites I find very helpful for getting information about \C++:
		\begin{itemize}
			\item \url{https://en.wikipedia.org/wiki/C\%2B\%2B11}
			\item \url{http://cppreference.com}
			\item \url{https://cppandbeyond.com}
			\item \url{https://stackoverflow.com}
			\item \url{https://isocpp.org}
			\item \url{http://meetingcpp.com}
		\end{itemize}
		
		\pause
		
		\emptyline
		The Standard document is obviously also a very good source. \\
		The (free) draft is mostly fine, too.
	\end{frame}
	
	\begin{frame}{Cool, new features of \C++11}
		\begin{small}
			\tableofcontents
		\end{small}
	\end{frame}
	
	\section{(rvalue-references)}
	\section{(initializer lists)}
	
	\section{\ttfamily auto}
	\begin{frame}
		\begin{center}
			\lstinline[basicstyle=\Huge\ttfamily]|auto|
			
			\emptyline
			{\Large type deduction for declarations}
		\end{center}
	\end{frame}
	
	\begin{frame}{\lstinline|auto|}
		When defining variables you can let \C++ \textbf{deduce the type} from the \textbf{initializing expression}.
		
		\only<beamer:0>{
			\emptyline
			\lstinline|auto| resolves only to \textbf{non-cv\footnote{\lstinline|const| and \lstinline|volatile|}-qualified value- and pointer-types}.
		}
	\end{frame}
	
	\begin{frame}[fragile=singleslide]{\lstinline|auto| examples}
		\begin{lstlisting}[gobble=12]
			auto a = 5;
			const auto b = 23;
			const auto &c = b;
			
			auto &d = b;        //error: mutable reference from const
			auto e = 0, f = 0u; //error: different types inferred
			auto g;             //error: nothing to deduce type from
		\end{lstlisting}
	\end{frame}
	
	\begin{frame}[fragile=singleslide]{\lstinline|auto| examples}
		\begin{lstlisting}[gobble=12]
			template<typename T>
			struct Foo
			{
			    typedef std::map<int, T> TypeAssocMap;
			    
			    static const TypeAssocMap& GetAssociations();
			};
		\end{lstlisting}
			
		\begin{lstlisting}[gobble=12]
			const auto &assocs = Foo<std::size_t>::GetAssociations();
			//assocs is of type const std::map<int, std::size_t>&
			//a.k.a. const Foo<std::size_t>::TypeAssocMap&
		\end{lstlisting}
	\end{frame}
	
	\begin{frame}[fragile]{\lstinline|auto| examples}
		\begin{lstlisting}
			auto iter = assocs.rbegin();
			//iter is of type std::reverse_iterator<implementation-defined>
			//  implementation-defined a.k.a.
			//  std::map<int, std::size_t>::const_iterator
			//a.k.a. std::map<int, std::size_t>::const_reverse_iterator
			//a.k.a. Foo<std::size_t>::TypeAssocMap::const_reverse_iterator
		\end{lstlisting}
		
		\pause
		
		\begin{lstlisting}
			const auto &elem = *iter;
			//elem is of type const std::pair<int, std::size_t>&
			//a.k.a. const std::map<int, std::size_t>::value_type&
			//a.k.a. const Foo<std::size_t>::TypeAssocMap::value_type&
		\end{lstlisting}
	\end{frame}
	
	\section{\ttfamily decltype}
	\begin{frame}
		\begin{center}
			\lstinline[basicstyle=\Huge\ttfamily]|decltype|
			
			\emptyline
			{\Large type of expressions and declared variables}
		\end{center}
	\end{frame}
	
	\begin{frame}{\lstinline|decltype|}
		With \lstinline|decltype| you can \textbf{deduce the type} of an expression or the \textbf{declared type} of a variable.
		
		\only<beamer:0>{
			\emptyline
			This includes cv-qualified value-, pointer- and reference-types.
		}
	\end{frame}
	
	\begin{frame}[fragile]{\lstinline|decltype| examples}
		\begin{lstlisting}
			volatile int a = 5, &b = a;
			decltype(b) c = b; //c is of type volatile int&
			auto        d = b; //d is of type          int
		\end{lstlisting}
		
		\pause
		
		\begin{lstlisting}
			//weird to express before
			template<typename T, typename U>
			decltype(T() - U()) foo(T a, U b) { return a - b; }
		\end{lstlisting}
		
	\end{frame}
		
	\begin{frame}[fragile]{\lstinline|decltype| examples}
		\begin{lstlisting}
			decltype(Foo<std::size_t>::GetAssociations()) assocs =
			    Foo<std::size_t>::GetAssociations();
			//assocs will be a const reference
			//more to type, but also more agnostic to changes
			//with our implementation
		\end{lstlisting}
		
		\pause
		
		\only<beamer:0>{
			Not too useful in a familiar environment, but when you want to write
			\textbf{generic, type agnostic algorithms} or such, \lstinline|decltype| can help immensely.
			
			Also
		}%
		\only<handout:0>{Useful}
		when just \textbf{declaring variables}, as \lstinline|auto| won't work there
		(nothing to infer the type from).
	\end{frame}
	
	\section{\ttfamily constexpr}
	\begin{frame}
		\begin{center}
			\lstinline[basicstyle=\Huge\ttfamily]|constexpr|
			
			\emptyline
			{\Large compile-time evaluation of functions}
		\end{center}
	\end{frame}
	
	\begin{frame}{\lstinline|constexpr|}
		Compile-time constants then:
		\begin{itemize}
			\item integral constants via \lstinline|enum| or \lstinline|const| (except \lstinline|extern|'d)
			\item literals \hint{\lstinline|\#define|}
			\item (most) expressions containing these
		\end{itemize}
		
		\pause
		
		\emptyline
		Now also single-statement functions (cannot return \lstinline|void|)
		
		and \textbf{constructors} (must have (more or less) empty body).
		
		\emptyline
		Variables can also be declared \lstinline|constexpr| \rarrow ensures ``compile-timeness''.
	\end{frame}
	
	\begin{frame}[fragile=singleslide]{\lstinline|constexpr| examples}
		\begin{lstlisting}[gobble=12]
			struct Foo
			{
			    constexpr Foo(std::size_t i):I(i) {}
			    
			    constexpr Foo operator+(std::size_t i) const
			        { return Foo(GetI() + i); }
			    
			    constexpr std::size_t GetI() const { return I; }
			    
			private:
			    std::size_t I;
			};
			
			constexpr std::size_t one = 1;
			char buf[(Foo(255) + one).GetI()];
		\end{lstlisting}
	\end{frame}
	
	\section{defaulting and deleting functions}
	\begin{frame}
		\begin{center}
			{\Huge defaulting and deleting functions}
			
			\emptyline
			{\Large implement trivial functions and remove functions}
		\end{center}
	\end{frame}
	
	\begin{frame}{defaulting and deleting functions - \lstinline|default|}
		\only<beamer:0>{
			\C++ can implicitly define the default-, copy- and move-constructors, destructors, \\
			and copy- and move-assignment operators.
			
			\emptyline
		}
		
		The compiler may not
		\only<beamer:0>{%
			do that (like writing a constructor prevents implicit definition of default-, copy- and move-constructors)%
		}%
		\only<handout:0>{%
			implicitly create a function%
		}%
		, \\
		but you can \textbf{explicitly default} them, in which case they will be defined as if they had been implicitly created.
	\end{frame}
	
	\begin{frame}[fragile=singleslide]{defaulting and deleting functions - \lstinline|default| examples}
		\begin{lstlisting}[gobble=12]
			struct Foo
			{
			    Foo();
			    Foo(const Foo&) = default;
			    virtual ~Foo();
			};
			
			Foo() { /*...*/  }
			Foo::~Foo() = default;
		\end{lstlisting}
	\end{frame}
	
	\begin{frame}[fragile=singleslide]{defaulting and deleting functions - \lstinline|default| examples}
		\begin{lstlisting}[gobble=12]
			struct Singleton
			{
			    static Singleton& Instance()
			        { static Singleton s; return s; }
			    
			private:
			    Singleton() = default;
			};
		\end{lstlisting}
	\end{frame}
	
	\begin{frame}{defaulting and deleting functions - \lstinline|delete|}
		You can also \lstinline|delete| functions with the same syntax.
		
		\emptyline
		\lstinline|delete| can:
		\begin{itemize}
			\item prevent implicit functions
			\item suppress inherited functions
			\item \lstinline|delete| overloads (to prevent implicit conversion)
		\end{itemize}
	\end{frame}
	
	\begin{frame}[fragile=singleslide]{defaulting and deleting functions - \lstinline|delete| examples}
		\begin{lstlisting}[gobble=12]
			struct Foo
			{
			    Foo(const Foo&) = delete;
			    int Bar();
			    int Bar(long);
			    int Bar(int) = delete;
			};
			struct Baz : Foo
			{
			    int Bar(long) = delete;
			};
			
			Foo f;
			Foo o(f);  //error: Foo(const Foo&) deleted
			f.Bar(1);  //error: Bar(int) deleted
			f.Bar(1l); //ok
			
			Baz b;
			b.Bar(1l); //error: Bar(long) deleted
			b.Bar();   //ok
		\end{lstlisting}
	\end{frame}
	
	\section{delegating constructors}
	\begin{frame}
		\begin{center}
			{\Huge delegating constructors}
			
			\emptyline
			{\Large invoking a constructor from another constructor}
		\end{center}
	\end{frame}
	
	\begin{frame}{delegating constructors}
		Sometimes you do very \textbf{similar initialization} from constructors with different parameters.
		
		\emptyline
		In \C++11 you can write one constructor to do most or all of that work and let \textbf{other constructors delegate} to it.
		
		\pause
		
		\emptyline
		The syntax is like calling a parent constructor.
		
		\only<beamer:0>{
			\emptyline
			You cannot do any further initialization.
			
			You can put code in the constructor body though, just nothing else in the initialization list.
		}
	\end{frame}
	
	\begin{frame}[fragile=singleslide]{delegating constructors examples}
		\begin{lstlisting}[gobble=12]
			struct Foo
			{
			    Foo():Foo(1, 1, 1) {}
			    
			    Foo(int a, int b, int c):A(a), B(b), C(c), Sum(A + B + C) {}
			    
			    template<typename Container>
			    Foo(const Container &c):Foo(c[0], c[1], c[2]) {}
			    
			    int A, B, C;
			    int Sum;
			};
		\end{lstlisting}
	\end{frame}
	
	\section{\ttfamily static\_assert}
	\begin{frame}
		\begin{center}
			\lstinline[basicstyle=\Huge\ttfamily]|static_assert|
			
			\emptyline
			{\Large compile-time assertions}
		\end{center}
	\end{frame}
	
	\begin{frame}<beamer:0>[fragile=singleslide]{\lstinline|static_assert| - runtime assert}
		We already have the \lstinline|assert|-macro in assert.h for run-time assertions.
		
		\emptyline
		One possible approach for a compile-time assertion was to conditionally allocate an array of size 0 or -1.
		
		\begin{lstlisting}[gobble=12,xleftmargin=2em]
			#define STATIC_ASSERT(cond) \
			    { typedef int _assert[(cond) ? 0 : -1]; }
		\end{lstlisting}
		
		This does not make for a precise error message.
		
		\emptyline
		\hint{There are better, but less smaller ways to do this in C++98.}
	\end{frame}
		
	\begin{frame}{\lstinline|static_assert|}
		\C++11 defines \lstinline|static_assert|, which takes a
		\only<beamer:0>{constant expression evaluating to}
		\lstinline|bool|%
		\only<beamer:0>{-expression}
		and a string literal as arguments.
		
		\pause
		
		\emptyline
		When the expression is true, nothing happens.
		
		When it is false, the compiler will error and display the string.
	\end{frame}
	
	\begin{frame}[fragile=singleslide]{\lstinline|static_assert| examples}
		\begin{lstlisting}[gobble=12]
			template<typename Float, typename Integer = int>
			Integer round2int(const Float f)
			{
			    static_assert(std::is_floating_point(f),
			        "Trying to round non-floating point type");
			    return static_cast<Integer>(f - static_cast<Float>(0.5));
			}
		\end{lstlisting}
	\end{frame}
	
	\begin{frame}[fragile=singleslide]{\lstinline|static_assert| examples}
		\begin{lstlisting}[gobble=12]
			struct RGB
			{
			    RGB(const RGB&) = default;
			    RGB(const uint8_t val[]):RGB(*reinterpret_cast<RGB*>(val)) {}
			    
			    operator uint8_t*()
			    {
			        return reinterpret_cast<uint8_t*>(this);
			    }
			    
			    uint8_t R, G, B;
			};
			static_assert(
			    std::is_standard_layout<RGB>::value,
			    "struct RGB must be of standard layout.");
		\end{lstlisting}
	\end{frame}
	
	\section{enum classes}
	\begin{frame}
		\begin{center}
			{\Huge enum classes}
			
			\emptyline
			{\Large scoped and stronger typed enums}
		\end{center}
	\end{frame}
	
	\begin{frame}{enum classes - traditional enums}
		In \C++, enumeration-types are one-way implicitly convertible:
		
		Enum-member to integers.
		
		\only<beamer:0>{
			You can still \lstinline|static_cast| integers to enum-types.
		}
	\end{frame}
	
	\begin{frame}[fragile=singleslide]{enum classes examples - traditional enums}
		\begin{lstlisting}[gobble=12]
			enum Day
			{
			    Monday, Tuesday, Wednesday,
			    Thursday, Friday, Saturday, Sunday
			};
			
			Day d = Sunday;
			int i = d + 2;  //okay
			i  = d / d;     //okay
			i *= d;         //okay
			++d;            //error: no operator++(Day)
			d += 2;         //error: no operator+=(Day, int)
			d  = d + d;     //error: cannot convert int to Day
			d  = -d;        //error: cannot convert int to Day
		\end{lstlisting}
	\end{frame}
	
	\begin{frame}[fragile=singleslide]{enum classes examples - traditional enums overload operators}
		\begin{lstlisting}[gobble=12]
			//do increment with wrap-around
			Day& operator++(Day &day)
			{
			    static_assert(sizeof(day) >= sizeof(int));
			    if(static_cast<int&>(day)++ == Sunday)
			    {
			        day = Monday;
			    }
			    return day;
			}
			
			//disallow arbitrary addition
			int operator+(Day&, int) = delete;
			//.. other integer-types, swapped parameters
		\end{lstlisting}
	\end{frame}
	
	\begin{frame}{enum classes}
		Enum classes are \textbf{not} implicitly convertible to integer.
		
		\only<beamer:0>{
			You can still \lstinline|static_cast| between these types.
		}
		
		\pause
		
		\emptyline
		Enum classes are scoped.
		
		\only<beamer:0>{
			\hint{\C++11 also allows scoped access to non-class enums.}
		}
		
		\pause
		
		\emptyline
		Enum classes support specifying an underlying type.
		
		\only<beamer:0>{
			This is done via a colon and the type after the enum class declaration.
			
			\hint{\C++11 also allows specifying one for non-class enums.}
		}
	\end{frame}
	
	\begin{frame}[fragile=singleslide]{enum classes examples}
		\begin{lstlisting}[gobble=12]
			enum class Day
			{
			    Monday, Tuesday, Wednesday,
			    Thursday, Friday, Saturday, Sunday
			};
			
			Day d = Sunday; //error: undeclared identifier 'Sunday'
			int i = d + 2;  //error: no operator+(Day, int)
			i  = d / d;     //error: no operator/(Day, Day)
			i *= d;         //error: no operator*=(int, Day)
			++d;            //error: no operator++(Day)
			d += 2;         //error: no operator+=(Day, int)
			d  = d + d;     //error: no operator+(Day, Day)
			d  = -d;        //error: no operator-(Day)
		\end{lstlisting}
	\end{frame}
	
	\begin{frame}[fragile=singleslide]{enum classes examples ``fixed''}
		\begin{lstlisting}[gobble=12]
			enum class Day
			{
			    Monday, Tuesday, Wednesday,
			    Thursday, Friday, Saturday, Sunday
			};
			
			Day d = Day::Sunday;
			int i = static_cast<int>(d) + 2;
			d = static_cast<Day>(2);
		\end{lstlisting}
		
		\texttt{\emptyline[3.29]}
	\end{frame}
	
	\begin{frame}[fragile=singleslide]{enum classes examples - overload operators}
		\begin{lstlisting}[gobble=12]
			//do increment with wrap-around
			Day& operator++(Day &day)
			{
			    //no static_assert -> enum class Day : int { ... }
			    if(static_cast<int&>(day)++ == Sunday)
			    {
			        day = Monday;
			    }
			    return day;
			}
			Day operator++(Day &day, int)
			{
			    Day prev = day;
			    ++day;
			    return prev;
			}
			
			//Day + int already not possible
			//due to no implicit enum class -> int cast
		\end{lstlisting}
	\end{frame}
	
	\section{explicit type conversion operators}
	\begin{frame}
		\begin{center}
			{\Huge explicit type conversion operators}
			
			\emptyline
			{\Large requiring \lstinline|static_cast| for type conversion}
		\end{center}
	\end{frame}
	
	\begin{frame}{explicit type conversion operators}
		\only<beamer:0>{
			\C++ type conversion operators convert types implicitly, \\
			like single-argument constructors can.
			
			\emptyline
		}
		We were already able to specify that constructors have to be invoked explicitly.
		
		Now we can do the same thing for type conversion operators.
	\end{frame}
	
	\begin{frame}[fragile=singleslide]{explicit type conversion operators examples - ctor}
		\begin{lstlisting}[gobble=12]
			struct Type
			{
			    Type() = default;
			    explicit Type(ExType);
			    Type(ImType);
			};
			
			void doType(Type);
		\end{lstlisting}
		
		\texttt{\emptyline[-0.2]}
		
		\begin{lstlisting}[gobble=12]
			ExType e; ImType i;
			Type f0 = i;                        //okay
			Type f1 = e;                        //error
			Type f2(e);                         //okay
			doType(i);                          //okay
			doType(e);                          //error
			doType(Type(e));                    //okay
		\end{lstlisting}
	\end{frame}
	
	\begin{frame}[fragile=singleslide]{explicit type conversion operators examples}
		\begin{lstlisting}[gobble=12]
			struct Type
			{
			    Type() = default;
			    explicit operator ExType();
			    operator ImType();
			};
			
			void doExType(ExType);
			void doImType(ImType);
		\end{lstlisting}
			
		\begin{lstlisting}[gobble=12]
			Type t;
			ImType  i = t;                      //okay
			ExType e0 = t;                      //error
			ExType e1 = static_cast<ExType>(t); //okay
			doImType(t);                        //okay
			doExType(t);                        //error
			doExType(static_cast<ExType>(t));   //okay
		\end{lstlisting}
	\end{frame}
	
	\section{\ttfamily final and \ttfamily override}
	\begin{frame}
		\begin{center}
			{\Huge \lstinline[basicstyle=\Huge\ttfamily]|final| and \lstinline[basicstyle=\Huge\ttfamily]|override| specifiers}
			
			\emptyline
			{\Large convey (and assert) information about the inheritance hierarchy}
		\end{center}
	\end{frame}
	
	\begin{frame}{\lstinline|final| and \lstinline|override| specifiers - \lstinline|final|}
		The \lstinline|final| specifier can be applied to classes and virtual functions.
		
		\emptyline
		Applied to a \\
		\leftskip=2em
		\begin{tabular}{l@{ \rarrow}l}
			virtual function & function cannot be overridden \\
			class            & cannot be inherited
		\end{tabular}
		
		\leftskip=0pt
	\end{frame}
	
	\begin{frame}[fragile=singleslide]{\lstinline|final| and \lstinline|override| specifiers - \lstinline|final| examples}
		\begin{lstlisting}[gobble=12]
			struct Foo
			{
			    virtual void DoFoo() final;
			};
			
			struct Bar final : Foo
			{
			    void DoFoo(); //error: Foo::DoFoo is declared final
			};
			
			struct Baz : Bar //error: Bar is declared final
			{
			};
		\end{lstlisting}
	\end{frame}
	
	\begin{frame}{\lstinline|final| and \lstinline|override| specifiers - \lstinline|override|}
		The \lstinline|override| specifier can only be applied to virtual functions.
		
		\emptyline
		If a function is specified as \lstinline|override|, it must override a method.
	\end{frame}
	
	\begin{frame}[fragile=singleslide]{\lstinline|final| and \lstinline|override| specifiers - \lstinline|override| examples}
		\begin{lstlisting}[gobble=12]
			struct Base
			{
			    virtual void Foo();
			    virtual void Bar(int);
				        void Baz();
			};
			
			struct D0 : Base
			{
			    int  Foo()     override; //error: return type differs
			    void Bar(long) override; //error: parameter types differ
			    void Baz()     override; //error: not virtual
			};
			
			struct D1 : Base
			{
			    void Foo()    const override; //error: cv-qualifiers differ
			    void Bar(int) override;       //okay
			    void Baz();                   //okay
			};
		\end{lstlisting}
	\end{frame}
	
	\section{lambda expressions}
	\begin{frame}
		\begin{center}
			{\Huge lambda expressions}
			
			\emptyline
			{\Large anonymous functions}
		\end{center}
	\end{frame}
	
	\begin{frame}[fragile]{lambda expressions}
		Expressions that create \textbf{anonymous\footnote{nameless} functions}.
		
		Actually more than a function: a \textbf{closure}.
		
		\pause
		
		Closure: You can enclose variables from the current scope.
		
		\pause
		
		\emptyline
		\lstinline[deletekeywords={return}]|[captures](parameters) -> return-type { body }|
		\pause
		\lstinline                         |[captures](parameters) { body }    | \hint{(return type deduced)}
		
		\pause
		
		\emptyline
		Lambdas without capture can be implicitly converted to a \textbf{function pointer}.
	\end{frame}
	
	\begin{frame}[fragile]{lambda expressions - capture}
		The ``capture'' of a lambda expression specifies the \textbf{variables to enclose}.
		
		\begin{itemize}
			\item \lstinline|[]  | no captures
			\item \lstinline|[x] | capture x by copy
			\item \lstinline|[&x]| capture x by reference
			\item \lstinline|[=] | capture all (by copy)
			\item \lstinline|[&] | capture all (by reference)
		\end{itemize}
		
		\pause
		
		\emptyline
		`Capture all' refers to all variables not already mentioned. \\
		Multiple captures are separated with a comma.
		
		\emptyline
		\only<beamer:0>{
			The ``capture all'' captures only the variables actually used in the lambdas body.
		}
	\end{frame}
	
	\begin{frame}[fragile]{lambda expressions examples}
		\begin{lstlisting}
			std::vector<int> list = MaekAwsumLst();
			int inc = 2, total = 0;
			std::for_each(list.begin(), list.end(),
			              [inc,&](int& x){ total += x; x += inc; });
		\end{lstlisting}
		
		\pause
		
		\begin{lstlisting}
			//std::qsort(void *ptr, std::size_t count, std::size_t size,
			//           int (*comp)(const void *, const void *));
			int32_t a[] = { 7, 5, 8, 7, 4, 6, 9 };
			std::qsort(a, 7, 4,
			           [](const void *pa, const void *pb) -> int
			           {
			               auto a = *static_cast<const int32_t*>(pa);
			               auto b = *static_cast<const int32_t*>(pb);
			               return a < b ? -1 :
			                      a > b ?  1 :
			                      0;
			           });
		\end{lstlisting}
	\end{frame}
	
	\section{member initialization}
	\begin{frame}
		\begin{center}
			{\Huge member initialization}
			
			\emptyline
			{\Large initialize members inline}
		\end{center}
	\end{frame}
	
	\begin{frame}{member initialization}
		\only<beamer:0>{Via member initialization you}%
		\only<handout:0>{You}
		can provide a \textbf{default value} for member variables%
		\only<beamer:0>{ that will be used if the constructor does not initialize them}.
		
		\pause
		
		\emptyline
		\only<beamer:0>{
			\hint{Although you can then default the default constructor, the constructor will not be trivial.}
		}
		
		Ordering is important.
		
		\only<beamer:0>{
			If you use other member-variables in the initialization they must be declared above it.
			
			It will compile otherwise, but upon a construction using it you are entering the domain of undefined behavior.
		}
	\end{frame}
	
	\begin{frame}[fragile=singleslide]{member initialization examples}
		\begin{lstlisting}[gobble=12]
			struct Foo
			{
			    Foo();
			    Foo(int a);
			    
			    std::string A = "forty";
			    std::string B = A + "two";
			    std::string C = B + " point zero";
			};
			
			//note: not trivial!
			Foo::Foo() = default;
			Foo::Foo(int a)
			    : A(std::to_string(a))
			    //B will be std::to_string(a) + "two"
			    , C(B)
			    //C will be std::to_string(a) + "two"
			{}
		\end{lstlisting}
	\end{frame}
	
	\section{raw string literals}
	\begin{frame}
		\begin{center}
			{\Huge raw string literals}
			
			\emptyline
			{\Large string literals without escape sequences}
		\end{center}
	\end{frame}
	
	\begin{frame}[fragile=singleslide]{raw string literals}
		Raw string literals are string-literals in the form
		
		\begin{lstlisting}[gobble=12,xleftmargin=2em]
			R"PREFIX(string)PREFIX"
		\end{lstlisting}
		
		where \textit{PREFIX} is any character but ` ' `(' `)' `\textbackslash' and
		\only<beamer:0>{the control characters for horizontal and vertical tab, form feed and newline}
		\only<handout:0>{some control characters}.
		
		\only<beamer:0>{
			\emptyline
			The \textit{PREFIX}-string may be at most 16 characters long.
		}
	\end{frame}
	
	\begin{frame}[fragile=singleslide]{raw string literals examples}
		\begin{lstlisting}[gobble=12]
			std::string lion = R"L(\rawr\)L";
			//lion == "\\rawr\\"
			
			std::string newline = R"(
			)";
			//newline == "\n"
			//(independent of OS-specific file newline, e.g. \r\n)
			
			std::string dosPrompt = R"(C:\>)";
			//dosPrompt == "C:\\>"
			
			std::string optQuotedStringRegex = R"X((\w+)|"([\w\s]+)")X";
			//optQuotedStringRegex == "(\\w+)|\"([\\w\\s]+)\")"
			
			std::string mixed = R"(raw)" "\nand"
			    R"(mix\n
			    xed)";
			//mixed == "raw\nandmix\\n\n\txed"
		\end{lstlisting}
	\end{frame}
	
	\section{range-based \ttfamily for}
	\begin{frame}
		\begin{center}
			{\Huge range-based \lstinline[basicstyle=\Huge\ttfamily]|for|}
			
			\emptyline
			{\Large syntactic sugar for container (range) iteration}
		\end{center}
	\end{frame}
	
	\begin{frame}{range-based \lstinline|for|}
		The range-based \lstinline|for| provides a shorter syntax for iteration.
		
		Its main usage is for \textbf{containers}, but anything that returns \\
		\lstinline|InputIterator|s for \lstinline|begin()| and \lstinline|end()| can be utilized.
	\end{frame}
	
	\begin{frame}[fragile]{range-based \lstinline|for| examples}
		\begin{lstlisting}[title=traditional \lstinline|for|]
			for(auto i = begin(foo); i != end(foo); ++i)
			{
			    i->bar(*i);
			}
		\end{lstlisting}
		
		\pause
		
		\begin{lstlisting}[title=range-based \lstinline|for|]
			for(auto &e: foo)
			{
			    e.bar(e);
			}
		\end{lstlisting}
	\end{frame}
	
	\begin{frame}[fragile]{range-based \lstinline|for| - custom iterator examples}
		\begin{lstlisting}[title=\lstinline|begin()| and \lstinline|end()| as members]
			struct Foo
			{
			    iterator begin();
			    iterator end();
			};
			
			for(auto x : Foo()){} //okay
		\end{lstlisting}
		
		\pause
		
		\begin{lstlisting}[title=\lstinline|begin()| and \lstinline|end()| as free functions]
			extern iterator begin(Foo&);
			extern iterator end(Foo&);
			
			for(auto x : Foo()){} //okay
		\end{lstlisting}
		
		\pause
		
		Remember to provide \lstinline|const| versions.
	\end{frame}
	
	\begin{frame}[fragile=singleslide]{range-based \lstinline|for| - Boost\footnote{reputable, peer-reviewed collection of \C++ libraries for general and various specific applications}.Range}
		Works amazingly well with Boost.Range!
		
		\emptyline
		\begin{lstlisting}[gobble=12]
			using namespace boost::adaptors;
			std::map<std::string, whatevs> someMap = ...;
			for(auto &x : someMap
			              //take elements [2..10)
			            | sliced(2, 10)
			              //only take keys from the map
			            | map_keys
			              //reverse the order (so elements (10..2])
			            | reversed
			              //split key into individual words
			            | tokenized(boost::regex(R"(\w+)"))
			            )
			{
			}
		\end{lstlisting}
	\end{frame}
	
	\begin{frame}<handout:0>{The End}
		\begin{center}
			\emptyline[2]
			{\Huge Thank you for tuning in.}
			
			\emptyline
			{\LARGE Questions?}
			
			\emptyline[10]
			\hint{{\large {\ttfamily :3}}}
			
			\hint{{\large *rawr*}}
		\end{center}
	\end{frame}
\end{document}
